\documentclass{article}
\usepackage{url}
\usepackage{fullpage}
\usepackage{graphicx}
\usepackage{hevea}

% Make links between references
\usepackage{hyperref}
\newif\ifpdf
\ifx\pdfoutput\undefined
  \pdffalse
\else
  \pdfoutput=1
  \pdftrue
\fi
\ifpdf
  \hypersetup{colorlinks=true, hyperindex=true, citecolor=red, urlcolor=blue}
\fi

\begin{document}

\title{Basic BioSQL with Biopython}
\author{Brad Chapman (chapmanb@arches.uga.edu), Michael Hoffman
(grouse@mail.utexas.edu)}

\maketitle
\tableofcontents

\section{Other BioSQL documentation}

This documentation provides basic descriptions of using BioSQL from
python. To keep it simple, it does not go into non-python specific parts
of the documentation. The other documentation which will be helpful for
using BioSQL from Biopython is currently available only from CVS. You
can get this documentation anonymously from the \verb|biosql-schema|
directory in the open-bio CVS servers. There are two ways to get these
docs:

\begin{enumerate}
\item You can use a commandline CVS client to get the code with the
following commands:

\begin{verbatim}
$ cvs -d :pserver:cvs@cvs.open-bio.org:/home/repository/biosql login
(Logging in to cvs@cvs.open-bio.org)
CVS password: <<PASSWORD is cvs (lowercase)>>
$ cvs -d :pserver:cvs@cvs.open-bio.org:/home/repository/biosql co
biosql-schema
\end{verbatim}

\item You can browse and download the code from the web interfaces to
CVS. The biosql CVS repository is located at:
\ahrefurl{\url{http://cvs.open-bio.org/cgi-bin/viewcvs/viewcvs.cgi/?cvsroot=biosql}}.
Click the \verb|Download tarball| link to get a \verb|tar.gz| file
containing the current CVS.
\end{enumerate}

Once you have gotten this code, there are two documents worth reading in
the \verb|doc| directory:

\begin{itemize}
  \item \verb|install.txt| -- This document contains basic information
  about BioSQL, installation instructions for the MySQL and PostgreSQL
  databases, and information about loading the BioSQL schema into a
  database.

  \item \verb|schema-overview.txt| -- This is a description of the
  current SQL schema making up the BioSQL databases.
\end{itemize}

\section{Installing python database packages}

In order to access databases from within python, you'll need to install
python interfaces to the database you are using. This section describes
the interfaces Biopython supports for using BioSQL and basic
installation.

These are designed with a UNIX-like platform in mind (Linux, any of the
various UNIXes, Mac OS X).

For all of these installations, you'll need to have at least a database
client installed, and also need to have the development libraries and
header files installed. On some systems that use packaging systems
(Red Hat, Debian, OS X) this means that you'll need to install a
development package (for example, mysql-dev or mysql-headers or 
mysql-libs) separately. If you start trying to install one of the python
interfaces below and start getting errors are missing libraries or
header files, you should check to make sure these are installed.

The \verb|install.txt| document mentioned above provides information on
install and configuring MySQL and PostgreSQL.

\subsection{MySQL}

MySQL access with python is fairly simple, as there is basically one
adaptor which is regularly developed and which most people use.

\subsubsection{MySQLdb}

The MySQLdb package is a library for accessing MySQL databases.
This interface is quite stable and completely DB API-2.0 compliant and
seems to definitely be the one to use for MySQL access. The latest
releases are available from: 
\ahrefurl{\url{http://sourceforge.net/projects/mysql-python}}.


To install, first download and unpack the tar.gz file:

\begin{verbatim}
$ tar -xzvpf MySQL-python-0.9.2.tar.gz 
MySQL-python-0.9.2/
MySQL-python-0.9.2/MySQLdb/
.....
\end{verbatim}

Then change into the directory containing the MySQLdb code:

\begin{verbatim}
$ cd MySQL-python-0.9.2
\end{verbatim}

MySQLdb uses the standard distutils mechanisms for installing programs,
so you should be able to build the code with:

\begin{verbatim}
$ python setup.py build
running build
running build_py
.....
\end{verbatim}

The \verb|setup.py| file is pretty good at finding the MySQL libraries
and header files on a number of platforms, so this should normally work
well (see platform specific notes below for special cases).

Finally, to install the library you'll need to be root on the machine,
and do:

\begin{verbatim}
#python setup.py install
running install
.....
\end{verbatim}

To test that everything worked smoothly, you should be able to do:

\begin{verbatim}
$ python
Python 2.2.2 (#1, 01/12/03, 07:51:34) 
[GCC Apple cpp-precomp 6.14] on darwin
Type "help", "copyright", "credits" or "license" for more information.
>>> import MySQLdb
>>>
\end{verbatim}

If this works happily, then you should be all set.
\\


\textbf{\large Platform specific notes:}
\begin{description}
  \item[Mac OS-X] -- With the default setup.py file, I get the error:

  \verb|ld: can't locate file for: -lmysqlclient_r|
  
  The setup.py file
  is apparently looking for a specially named thread-safe library, which
  I didn't have installed by default. To fix this you can either
  install the latest mysql-devel from
  \ahrefurl{\url{http://www.mysql.com/}}, or edit MySQLdb's
  \verb|setup.py| file, changing this line:

  \verb|thread_safe_library = YES|

  to:

  \verb|thread_safe_library = NO|
  
\end{description}

\subsection{PostgreSQL}

In contrast to MySQL, PostgreSQL has a mess of python interfaces all of
which are in varying states of completion and maintenance. The most
frustrating problem with many of these is that they do not necessarily
support the DB API-2.0 which makes them very difficult to use with
standard python SQL access code.


Below are the currently supported interfaces. More could of course be
added, but probably won't be unless someone gets very excited about
them.

\subsubsection{psycopg} 

psycopg is a nice PostgreSQL python interface developed along with the
Zope project. It is available from 
\ahrefurl{\url{http://www.zope.org/Members/fog/psycopg}}, and is fully
DP API-2.0 compliant.


psycopg requires you to make the mxDateTime package installed for
handling of dates. This comes in the mxBase set of packages which are
also required for Biopython, and you can get it from
\ahrefurl{\url{http://www.egenix.com/files/python/eGenix-mx-Extensions.html\#Download-mxBASE}}.


To begin installing, psycopg download the tar.gz file, unpack it, and
change into the new directory created:

\begin{verbatim}
$ tar -xzvpf psycopg-1.0.14.tar.gz 
psycopg-1.0.14/
psycopg-1.0.14/AUTHORS
....
$ cd psycopg-1.0.14
\end{verbatim}

psycopg uses the autoconf build system, so it follows a
configure/make/make install set of steps.

First, run the configuration:

\begin{verbatim}
$ ./configure
creating cache ./config.cache
checking for python... /sw/bin/python
checking python version... 2.2
....
\end{verbatim}

If the configure has trouble finding the postgreSQL or mxDateTime
include files ans libraries, a few useful flags to configure are:

\verb|--with-postgres-libraries=DIR|,

\verb|--with-postgres-includes=DIR|,

\verb|--with-mxdatetime-includes=DIR|. 

For example:

\begin{verbatim}
$ ./configure --with-mxdatetime-includes=/sw/lib/python2.2/site-packages/mx/DateTime/mxDateTime
\end{verbatim}

is necessary on Mac OS X since the mxDateTime header files are in a
unusual location.

After the configure finished happily, you need to make the module:

\begin{verbatim}
$ make
gcc  -DNDEBUG -O3 -Wall -Wstrict-prototypes  -I/sw/include/python2.2
  -I/sw/lib/python2.2/config -DHAVE_CONFIG_H=1  -DHAVE_LIBCRYPTO=1
  -DHAVE_ASPRINTF=1  -I/sw/include/postgresql -I/sw/include/postgresql/server
  -I/sw/lib/python2.2/site-packages/mx/DateTime/mxDateTime
-DVERSION=\"1.0.14\" -DNDEBUG -D_REENTRANT -D_GNU_SOURCE   -c
././module.c -o ./module.o
....
\end{verbatim}

Finally, become root and install the module:

\begin{verbatim}
# make install
Installing shared modules...
install -m 555 ./psycopgmodule.so /sw/lib/python2.2/site-packages
\end{verbatim}

To test that everything worked okay, fire up python and import the newly
created module:

\begin{verbatim}
$ python
Python 2.2.2 (#1, 01/12/03, 07:51:34) 
[GCC Apple cpp-precomp 6.14] on darwin
Type "help", "copyright", "credits" or "license" for more information.
>>> import psycopg
>>> 
\end{verbatim}

No error messages here and you've done everything correctly.
\\


\textbf{\large Platform specific notes:}
\begin{description}
  \item[Mac OS-X] -- I had a couple of things I needed to change to get
  psycopg to compile happily. First the make complains with:

  \verb|././module.c:156: only 1 arg to macro 'Dprintf' (2 expected)|

  There seems to have been kind of mistake in how this macro was defined
  for gnu C compilers. To fix the problem, I had to change line 107 in
  \verb|module.h| from:

  \verb|#define Dprintf(fmt, args...)|

  to:

  \verb|static void Dprintf(const char *fmt, ...) {}|

  After this the make will proceed for a while, and then complains with:

  \verb|ld: can't open: /sw/bin//sw/bin/python|

  This appears to be a glitch in how the Makefile is created. I had to
  change line 136 from:

  \verb|PYTHON = /sw/bin/python|

  to:

  \verb|PYTHON = python|

  After these two changes, everything compiled fine.
\end{description}

\section{Getting Started}

Once you've got your python database interface installed you're ready to
get started with BioSQL. This section is intented to get you going
with the basics in python. For in-depth information about the BioSQL
schema itself, you should check out the standard BioSQL documentation
which has extensive information about setting up databases, loading SQL
and describing what exactly is going on with all those SQL tables. For
specific Biopython use cases, see the cookbook items (hopefully to be
written) below.

\subsection{Prerequisites}

Allow of the work in this section assumes that you have installed a
database, a python binding to this database, and loaded the BioSQL
schema into the database. 


Additionally, this is written with the assumption that you know a little
about databases in general. The most important thing that is really
assumed is that you understand the client/server model of databases. The
basic idea is that there is a database server which actually physically
contains the data you are working with. This server can be located on
your machine, a machine across the room from you, or some computer
half-way around the world. We will not be dealing with the server here
-- we will be dealing with a client that connects to this server. 


Understanding this, for this example we are going to assume we have a
server/client set up with the following information:

\begin{description}
  \item[username] -- This is the user you are allowed to connect to the
  database server as. In this example, the username is:
  \textbf{chapmanb}

  \item[password] -- This is the password associated with the above use.
  For our example, this is: \textbf{biopython}

  \item[host] -- This is the hostname where the server computer is
  located. As mentioned above, this could be anywhere, but for this
  example we will use: \textbf{localhost}

  \item[db] -- The name of the database where the BioSQL schema has been
  loaded. For our example, this is: \verb|bioseqdb|

  \item[driver] -- The python database adaptor that you are using. The
  available drivers and their installation are described above.
  Obviously, the driver and your client have to match the database
  server type you will be dealing with. In our example, we are using a
  MySQL database with the driver: \textbf{MySQLdb}
  
\end{description}

To run these examples on your own, you'll need to know all of this
information and adjust what is written below to get it to work for you.

\subsection{Connecting to a BioSQL database}

This section describes the basic steps to connect with a BioSQL
database. Right now we are not assuming the database needs to be loaded
with any information (that comes next), but rather just enumerating the 
very basic steps for connecting with the database with the biopython
implementation of BioSQL.


Okay, now that all the preliminary ramblings are out of the way, let's
get started actually doing this. First, we load the Biopython
implementation of BioSQL:

\begin{verbatim}
>>> from BioSQL import BioSeqDatabase
\end{verbatim}

Now the second and final step -- connect with the database:

\begin{verbatim}
>>> server = BioSeqDatabase.open_database(driver = "MySQLdb", user = "chapmanb",
...             passwd = "biopython", host = "localhost", db = "bioseqdb")
\end{verbatim}

There you go -- you've got a connection. Hmmm, wonder why I did so much
rambling for such a simple thing? So does my psychologist.


Now that we've got the connecting to the database out of the way, let's
do some useful work with this database.

\subsection{Loading a GenBank file into the database}

Now that we've got a connection to our database, the next logical thing
to do is load some information into it. For this example, we are going
to assume we have a GenBank file on our computer called
\verb|cor6_6.gb| that we are going to work with. This is just a file of
GenBank sequences, and in this case contains a bunch of cold resistance
related genes from various plants. 


We want to load this file into the BioSQL server so we can query it.
The first thing we want to do is create a BioSQL database on the server
to cold this information. In this case, we'll simply call it
\verb|cold|. Again, this is a one-liner:

\begin{verbatim}
>>> db = server.new_database("cold")
\end{verbatim}

Now we want to do the loading of the file into the database. The
Biopython implementation works by taking a standard Iterator object that
returns Biopython SeqFeature objects and then doing the loading. If that
last sentence doesn't make any sense at all, then you should go check
out the standard Biopython documentation which explains it all in
excruciating detail. So, we need to set up our Iterator for our GenBank
file, which we can do with the following code:

\begin{verbatim}
>>> from Bio import GenBank
>>> parser = GenBank.FeatureParser()
>>> iterator = GenBank.Iterator(open("cor6_6.gb"), parser)
\end{verbatim}

With this iterator, the loading of the database is another one-liner:

\begin{verbatim}
>>> db.load(iterator)
6
\end{verbatim}

And the GenBank file is loaded into the database. Notice that the load
function returns the number of records loaded (6 in this case). This is
useful for sanity checking to make sure that you didn't try to load a
massive file and end up with a result like 3.

\subsection{Retrieving SeqRecord objects}

Now that our database is loaded with information, the next logical step
is retrieving information from that database. Let's start with our
initial connection to the database server that we got when we connected
to it above. The first step is getting a direct connection to our
\verb|cold| database. The server is set up to act like a python
dictionary, so we get our \verb|cold| database with the following code:

\begin{verbatim}
>>> db = server["cold"]
\end{verbatim}

Now that we've got the database, we can retrieve a record based on
accession numbers. So, to get a record for the Arabidopsis kin2 gene
(contained in the \verb|cor6_6.gb| file), we simply do:

\begin{verbatim}
>>> record = db.lookup(accession = "X62281")
\end{verbatim}

The \verb|record| that we get back models as exactly as possible a
standard SeqRecord object (again, referring to the Biopython
documentation is a great idea if you don't understand these standard
Biopython objects). So we can do things like retrieve basic information
about this sequence in the standard ways:

\begin{verbatim}
>>> record.id
'X62281'
>>> record.name
'ATKIN2'
>>> record.description
'A.thaliana kin2 gene.'
\end{verbatim}

We can deal with the sequence:

\begin{verbatim}
>>> sequence = record.seq
>>> sequence.alphabet
IUPACUnambiguousDNA()
>>> sequence[:5].tostring()
'ATTTG'
\end{verbatim}

We can deal with features:

\begin{verbatim}
>>> feature = record.features[0]
>>> feature.type
'source'
>>> print feature.location
(0..880)
>>> feature.qualifiers
{'strain': ['ssp. L. Heynh, Colombia'], 'organism': ['Arabidopsis
thaliana'], 'db_xref': ['taxon:3702']}
\end{verbatim}

And so on and so in. The basic idea is that you should be able to deal
with a sequence with annotations stored in a database in exactly the
same way you deal with sequences in flatfiles. All of the database
access happens under the hood so you don't even have to think about SQL
or other nastiness, but you get all the advantages of a database.


\section{Python Cookbook Code}

Doing various fun things to demonstrate the usefulness of BioSQL. I
suppose I could go ahead and write something here.

\end{document}
