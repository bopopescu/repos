\documentclass{beamer}


\usepackage{default}
\title{QC parameter tuning}
\subtitle{Choose an optimized 6-parameter setting for final genotype calls after oligo.}
\author{Yu Huang}
\institute{MCB USC}
\date{May 14, 2008}

\begin{document}

\setbeamertemplate{caption}
{
\begin{beamercolorbox}{}
     \vskip3pt \insertcaption \vskip3pt
  \end{beamercolorbox}
}

\frame{\titlepage}

\begin{frame}{Definition of strain}

\begin{quotation}
 


 1. Race; stock; generation; descent; family.
 [1913 Webster]

 He is of a noble strain. --Shak.
 [1913 Webster]

 With animals and plants a cross between different
 varieties, or between individuals of the same
 variety but of another strain, gives vigor and
 fertility to the offspring. --Darwin.
 [1913 Webster]
\end{quotation}
\end{frame}

\section{Genotype calling pipeline}
\begin{frame}{Genotype calling pipeline}
The 1st reference dataset is the one merging 2010, 149SNP data and 384-illumina (written in the order of overriding priority) with the number of SNPs spanning from 149 to 30,000 in 295 strains.

The 2nd reference dataset is the 20-strain whole-genome resequencing perlegen dataset.

\begin{enumerate}
 \item oligo genotype calling (parameter learning using 31 arrays with correspondent perlegen calls)
 \item filter strains according to mismatch rate from a comparison with the 1st reference dataset
 \item filter strains according NA rate
 \item filter snps according to mismatch rate from a comparison with the 2nd reference dataset
 \item filter snps according to NA rate

\end{enumerate}
\end{frame}

\begin{frame}{Genotype calling pipeline (cont.)}
\begin{enumerate}

 \item merge the data with 2nd reference dataset. as far as the 2nd reference dataset has non-NA value,  substitute call at the corresponding strain/snp with this non-NA value.
 \item remove monomorphic snps
 \item (QC before imputation)
 \item imputation by NPUTE
 \item (QC after imputation)
\end{enumerate}
\end{frame}

\begin{frame}{goal of Genotype calling pipeline}
Choose the best setting of six parameters below:

min call probability (step 1.1), max strain/call mismatch rate (step 1.2), max strain/call NA rate (step 1.3), max snp mismatch rate (step 1.4), max snp NA rate (step 1.5), npute-window-size (step 2.5)

to obtain the objective of minimizing (strain/snp mismatch rate, no of strains/snps removed).
\end{frame}

\section{NPUTE window size matters little}
\begin{frame}{NPUTE window size matters little}

\begin{figure}
\includegraphics[height=0.7\textheight]{figures/figure2.png}
\caption{strain after imputation, max call NA rate=0.4, max snp mismatch rate=0.25, max snp NA rate=0.4, npute window size=50, X=max call mismatch rate, Y=npute window size, Z=avg mismatch rate.}\label{f2}
\end{figure}

\end{frame}

\begin{frame}{NPUTE window size matters little (cont)}
\begin{figure}
\includegraphics[height=0.7\textheight]{figures/figure1.png}
\caption{X-axis is NPUTE window size from 10 to 90. Y axis is average strain mismatch rate. Each horizontal 5 points correspond to a max accession mismatch rate cutoff.}\label{f1}
\end{figure}
\end{frame}

\section{choose min call probability + max strain mismatch rate}
\begin{frame}{choose min call probability + max strain mismatch rate}

\begin{figure}
\includegraphics[height=0.7\textheight]{figures/figure3.png}
\caption{strain or snp=strain, after imputation=1, max call NA rate=0.4, max snp mismatch rate=0.2, max snp NA rate=0.4, npute window size=50. min call probability, max strain mismatch rate, avg strain mismatch rate}\label{f3}
\end{figure}
\end{frame}

\begin{frame}{choose min call probability + max strain mismatch rate}
\begin{figure}
\includegraphics[height=0.7\textheight]{figures/figure4.png}
\caption{strain or snp=strain, after imputation=1, max call NA rate=0.4, max snp mismatch rate=0.2, max snp NA rate=0.4, npute window size=50, min call probability, max call mismatch rate, no of total accessions filtered}\label{f4}
\end{figure}
\end{frame}

\begin{frame}{choose min call probability + max strain mismatch rate}
\begin{figure}
\includegraphics[height=0.7\textheight]{figures/figure5.png}
\caption{strain or snp=strain, after imputation=1, max call NA rate=0.4, max snp mismatch rate=0.2, max snp NA rate=0.4, npute window size=50, min call probability, max call mismatch rate, no of total snps removed}\label{f5}
\end{figure}
\end{frame}

\end{document}
