\documentclass[a4paper,10pt]{article}


%opening
\title{Paper}
\author{Yu Huang}

\begin{document}

\maketitle

\begin{abstract}

\end{abstract}

\section{Introduction}
Arabidopsis thaliana is a highly selfing species, with a very small portion resulting from outcrossing. Hartl et al\cite{Hartl2007} put the outcrossing rate to be around 1\%.

Given the polymorphism data of 149 SNPs in 4664 strains, we are able to estimate the outcrossing rate.

\section{Data}
The polymorphism data is collected by Justin Borevitz lab from University of Chicago. The affymetrix genotyping array is used. The 149 SNPs were picked according to the de-novo sequencing data\cite{Nordborg2005}. Figure~ shows the spacing of 149 SNPs. The probes are 25-mers with the changing middle base to detect two alleles. The 4664 strains were collected from around the world (Figure~ ).

\section{Genotyping Error Rate}
The 96 strains from \cite{Nordborg2005} were genotyped again, which offered an opportunity to have a rough estimate of genotyping error rate by comparing the data from two different experiments. Figure~ is an overview of all the differences.

The optimistic error rate (not counting the NAs) is 121/(12408+121) = 0.96\%. Taking the NAs into account, the error rate would be (121+893+804)/( 12408.0+121+893+804) = 12.78\%.


\section{Bogus Heterozygous Calls(column-wise)}


\section{model for snp locus}
\begin{equation}
P({SNP}_j|Strain heterozygous info) = \prod_{i=1}^{N} p_i^{a_i^j} (1-p_i)^{1-a_i^j}
\end{equation}


$p_i$ is probability that one strain has heterozygous call.


$a_i^j$ is indicator whether ${SNP}_j$ is homozygous($=1$) or not($=0$) for ${Strain}_i$.


\section{model to estimate the number of selfing generations since last outcrossing event}
\begin{equation}
D_i = 1 - \sum_{j=1}^k p_{ij}^2
\end{equation}


$p_{ij}$ is probability of allele j of SNP i. k is the number of alleles.

$D_i$ is probability of SNP i being heterozygous.


\begin{equation}
likelihood(S_n) = \prod_{i=1}^L {(\frac{D_i}{2^n})}^{a_i} {(1-\frac{D_i}{2^n})}^{1-a_i}
\end{equation}

\bibliography{outcrossing}
\bibliographystyle{plain}

\end{document}
